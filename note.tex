# primero para crear el modulo  py -m venv venv i crear la carpeta venv , es como npm x create--react " desde la terminal
para arrancarlo  
#para activar el entorno C:\Users\david\Desktop\django\venv\Scripts\Activate.ps1
en el caso de que no funciona con lo del f1 tienes que hacerlo manual 
# luego activamos django  pip install django dentro de la terminal para instal django
#para ver si se i django-admin --version , para ver si fue existoso 
# para crear la carpeta para el projecto django-admin startproject djandocrud . ==, el . es para que no haga la carpeta dentro de venv

#python manage.py runserver  == run para empezar a usar el proyecto en el localhost 8000

#para crear la base de datos sql . que se pueda ver en pantalla = python manage.py runserver lo hace en el localhost 8000
es como npm start etc

# para crear el overflow    python manage.py startapp tasks , para crear el entorno 
# luego ve a la carpeta principal que seria djangocrud i en INSTALLED_APP pon INSTALLED_APPS  en settings = [
    'django.contrib.admin',
    'django.contrib.auth',
    'django.contrib.contenttypes',
    'django.contrib.sessions',
    'django.contrib.messages',
    'django.contrib.staticfiles',
    'tasks'
]

#luego en task views crea un def para probar si todo va bien 
# importa from django.http import HttpResponse para crear el enlace 
def helloword(request):
 return HttpResponse('hello david')

 ahora en la carpta django deveria de estar la url urlpatterns = [
    path('', views.helloword), == que sea con un texto vacio
]
 en las url importa from tasks import views  i llama a path('',views.helloword) urlpatterns = [
    path('admin/', admin.site.urls),
    path('',views.helloword)
]

ahora para arrancar el proyecto python manage.py runserver

# crea una carpta en taks llamada templates para poner dentro el archibo html 
 ahora que cree la carpeta template con en eachivo html 

 cambia el formato para estraer views  
 from django.shortcuts import render
# from django.http import HttpResponse este lo elimino 
def helloword(request):
 return render(request, 'singup.html')


 #ahora vamos a crear el formulario , pero vamos a usar una biblioteca que nos da py
 from django.contrib.auth.forms  import UserCreationForm 
  esto es para crear el formulario de crear un usuario 

  def helloword(request):
 return render(request, 'singup.html'), {
    'form' : UserCreationForm()
 } 

 recuarda para mostarlo en pantalla vincularlo al singup {{form}}


 {{form.as_p}} == con esa propiedad hace que sea uno a cada lado


 ahora en el form creo una dirrection 
 <form action="/signup/" method="POST">
    {{form.as_p}}
    <button >
        SigUp 
    </button>
</form> LO DE ACTION ES PARA QUE ENVIE LOS DATOS I LO DE POST EN PARA QUE NO SE VEA LO QUE ENVIE 

path('signup/', views.helloword), crea un nuevo pach 

en el formulario anade 

para la seguridad de los token 
<form action="/signup/" method="POST">
    
    {{form.as_p}}
    <button >
        SigUp 
    </button>
</form>


ahora vamos a crear el signup en el html i crea el paht 
path('signup/', views.signup, name='signup'),
i la funcion 
def signup(request):
    return render(request,'signup.html', {
        'form': UserCreationForm
    })
ahora lo que quiero es el la funcion signup es ver si se esta enviado 
def signup(request):
    if request.method == 'GET':
        print('sending formulario')
    else:
        print('gettind data')
    return render(request,'signup.html', {
        'form': UserCreationForm
    })

ahora voy a crealo de forma cottecta PARA GUARDALOS


def signup(request):
    if request.method == 'GET':
        return render(request, 'signup.html')
    else:
        if request.POST['password1'] == request.POST['password2']:
            try:
                # Crear el usuario
                user = User.objects.create_user(
                    username=request.POST['username'],
                    password=request.POST['password1']
                )
                user.save()  # Guardar el usuario en la base de datos
                return HttpResponse('User created successfully')
            except:
                return HttpResponse('User already exists')
        return HttpResponse('Passwords do not match')

python manage.py migrate para   = para las migracion me va a crear las tablas 



def signup(request):
    if request.method == 'GET':
        # Pasamos un formulario vacío a la plantilla
        form = UserCreationForm()
        return render(request, 'signup.html', {'form': form})
    else:
        # Proceso de creación de usuario y verificación de contraseña
        form = UserCreationForm(request.POST)  # Reconstruimos el formulario con los datos enviados
        if request.POST['password1'] == request.POST['password2']:
            try:
                user = User.objects.create_user(
                    username=request.POST['username'],
                    password=request.POST['password1']
                )
                user.save()
                return HttpResponse('User created successfully')
            except:
                return render(request, 'signup.html', {
                    'form': form,
                    'error': 'User already exists'
                })
        else:
            return render(request, 'signup.html', {
                'form': form,
                'error': 'Passwords do not match'
            })
 de esta forma es de la fomar para crear el formulario

 #ahora voy a crear las tareas

 crea la funcion en views 
 def tasks(request):
    return render(request, 'tasks.html')

    path('tasks/', views.tasks, name='tasks') i ahora crea el path

    from django.shortcuts import render , redirect para que me lleve al path taks cuando se creo el usuario

    return redirect('tasks')

    ahora no se vera en las cookis importamos otra biblioteca 
    from django.contrib.auth import login

    tiene dos parametrod login(request,user)

    mira si se creo la cooquie en aplication cookies

    ahora vamos a depurar los errores con 
#from django.db import IntegrityError

ahora voy a crear el template base 
i dentro tienes que poner 

    
    

    en home add 
 para estraer 

 ahora en base voy a crear una  voy a crear la nav para que me dirija a taks or singup

  para importar la plantilla base que es la nav por ahora 

